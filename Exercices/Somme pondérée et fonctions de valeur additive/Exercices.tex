\documentclass[a4paper, 12pt]{article}
\usepackage{amsmath,mathtools}
\usepackage[utf8]{inputenc}

% Math commands
\newcommand*{\indif}[4]{#1 \prescript{}{#2}{\sim}_{#3} #4}

\renewcommand{\partname}{} 

\begin{document}
	\section*{Notation}
	Soit $i,j \in \mathbf{N}$, $c_i$ le critère d'indice $i$, $v_{c_i}$ la valeur du critère d'indice $i$,
	la notation:
	$$\indif{(v_{c_i},v'_{c_i})}{c_i}{c_j}{(v_{c_j},v'_{c_j})}$$
	Signifie que pour passer de $v_{c_i}$ à $v'_{c_i}$ sur le critère $c_i$, on est prêt à passer de
	$v_{c_j}$ à $v'_{c_j}$ sur le critère $c_j$.\\
	
	Par exemple, $\indif{(1,2)}{1}{2}{(5,4)}$ signifie que pour passer de $1$ à $2$ sur
	le critère $1$, on est prêt à passer de $5$ à $4$ sur le critère $2$.
	\section*{Exercice 1}
		\subsection*{Partie 1}
			Voir fichier \textit{ChoixDeThierry}, feuille 
			\textit{Somme pondérée} pour le travail Excel...\\
			%
			~\\
			%
			\textit{Norm-décal} représente une addition dans la
			 normalisation\\
			\textit{Norm-Divis} représente une division dans la 
			 normalisation
			%
		%
		\subsection*{Partie 2}
			On a effectué une transformation affine pour passer de
			$u_j(g_i)$ à $u^{'}_{j}(g_i)$. \\
			En effet,  $u^{'}_{j}(g_i) = 
			\frac{u_j(g_i)}{\max_j (u_j(g_i))}$\\
			%
			~\\
			%
			Par la normalisation, on a que 
			$$ k_1 \cdot \frac{\Delta \text{coût}}{\max \text{coût}} 
			 = k_2 \cdot \frac{\Delta \text{accel}}{\max \text{accel}}$$
			Choisissons 17000 comme coût de référence pour un delta de 1
			seconde d'accélération : 
			$$ -1 (\frac{17000}{21334} - \frac{x}{21334})
			 = -2 (\frac{28}{30.8) - \frac{29}{30.8}} $$
			$$ \Leftrightarrow 
			   (17000 - x) = 2 \cdot 21334 \cdot \frac{-1}{30.8} 
			 = -1385.32 $$
			Pour notre modèle, on aura donc une équivalence entre deux
			voiture différant de 1 sec d'accélération Ssi leurs prix 
			diffèrent de 1385.32 en coût.
			%
		%
	%
	\section*{Exercice 3}
		Soient 3 critères à \textbf{maximiser}.\\
		%
		~\\
		%
		\textbf{(1)} Supposons que le décideur nous dise que :
		$$ (2, 3, 4) \succ (4, 3, 3) $$
		On peut en déduire le modèle d'utilité suivant :
		$$ k_1 \cdot u^{'}_{1}(2) + k_2 \cdot u^{'}_{2}(3) + 
		   k_3 \cdot u^{'}_{3}(4) >
		   k_1 \cdot u^{'}_{1}(4) + k_2 \cdot u^{'}_{2}(3) + 
		   k_3 \cdot u^{'}_{3}(3) $$
		$$ \Leftrightarrow 
		   k_1 \cdot u^{'}_{1}(1) + k_3 \cdot u^{'}_{3}(4) >
		   k_1 \cdot u^{'}_{1}(4) + k_3 \cdot u^{'}_{3}(3) $$
		$$ \Leftrightarrow 
		   k_1 (u^{'}_{1}(1) - u^{'}_{1}(4)) > 
		   k_3 (u^{'}_{3}(4) - u^{'}_{3}(4)) $$
		$$ \Leftrightarrow 
		   (2, 4)\ _1>_3\ (3, 4) $$
		%
		\\~\\
		%
		\textbf{(2)} Et si le décideur nous dit que 
		$$ (2, 1, 4) \prec (4, 1, 3) $$
		On peut en déduire le modèle d'utilité suivant :
		$$ k_1 \cdot u^{'}_{1}(2) + k_2 \cdot u^{'}_{2}(1) + 
		   k_3 \cdot u^{'}_{3}(4) <
		   k_1 \cdot u^{'}_{1}(4) + k_2 \cdot u^{'}_{2}(1) + 
		   k_3 \cdot u^{'}_{3}(3) $$
		$$ \Leftrightarrow 
		   k_1 \cdot u^{'}_{1}(2) + k_3 \cdot u^{'}_{3}(4) <
		   k_1 \cdot u^{'}_{1}(4) + k_3 \cdot u^{'}_{3}(3) $$
		$$ \Leftrightarrow 
		   k_1 \cdot (u^{'}_{1}(1) - u^{'}_{1}(4)) < 
		   k_3 \cdot (u^{'}_{3}(4) - u^{'}_{3}(4)) $$
		$$ \Leftrightarrow 
		   (2, 4)\ _1<_3\ (3, 4) $$
		%
		Par \textbf{(1)} et \textbf{(2)}, on trouve qu'on ne peut pas
		utiliser le modèle par fonction de valeur
		additive, on retrouve le même problème que dans l'exemple 
		"frites mayo".
	%
	\section*{Exercice 4}
		\subsection*{Partie 1}
			(1, 5, 3) $\sim$ (2, 4, 3) $\rightarrow$ On remarque que le 
				$3^{\text{ème}}$ critère est commun aux deux vecteurs, 
				il vaut 3. On peut donc faire le rapport entre k1 et k2.
				$$ k_1\cdot u^{'}_{1}(1) + k_2\cdot u^{'}_{2}(5) + 
				k_1\cdot u^{'}_{3}(3) = k_1\cdot u^{'}_{1}(2) + 
				k_2\cdot u^{'}_{2}(4) + k_3\cdot u^{'}_{3}(3) $$
				$$ \Leftrightarrow 
				k_1\cdot u^{'}_{1}(1) + 
				k_2\cdot u^{'}_{2}(5) = k_1\cdot u^{'}_{1}(2) + 
				k_2\cdot u^{'}_{2}(4) $$
				$$ \Leftrightarrow 
					\frac{k_1}{k_2} = 
					\frac{u^{'}_{1}(2) - u^{'}_{1}(1)}
						 {u^{'}_{2}(5) - u^{'}_{2}(4))}$$
		\subsection*{Partie 2}
			En remettant les nouvelles valeurs dans la formule donnée 
			ci-dessus, on trouve:
			$$ \frac{k_1}{k_2} = \frac{4 - 2}{17 - 16} = \frac{2}{1}$$
		\subsection*{Partie 3}
			
\end{document}